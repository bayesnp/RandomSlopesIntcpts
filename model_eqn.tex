% !TEX TS-program = xelatex
% !TEX encoding = UTF-8 Unicode
%
% Above directives are recognized by TeXShop.  See detailed explanation
% on these and more:
% https://tex.stackexchange.com/questions/78101/when-and-why-should-i-use-tex-ts-program-and-tex-encoding
%
%%%% Place this as the first line to force TeXShop to save file as UTF-i
\documentclass{article}
\usepackage{amsmath}     % higher math, aligned math equations
\usepackage{mathtools}   % aligned columns in bracketed \bmatrix in amsmath
\usepackage{amssymb}
\usepackage{bm}          % bm gives you nice-looking \boldsymbol{y}
\usepackage{hyperref}    % allow ~ in \url{} better than \usepackage{url}
\usepackage{graphicx}    % graphicx allows you to define several 
% directories to be searched for figures. Each path has to end 
% with a / and be enclosed in curly braces { } even if only 
% one path is specified. Only the specified directories will 
% be searched and not their subdirectories.
\graphicspath{{\string~/resh/ahles/apoe/allostatic/icctf23/tex/figs/}{\string~/resh/ahles/apoe/allostatic/R.salth/}{./figures/}}
% NOTE: \string~ instead of simple ~

\title{Bayesian Varying Intercepts and Slopes Modeling of Cognitive Aging Trajectories in Breast Cancer Survivors as Compared to Matched Controls}
\author{Yuelin Li$^1$ $^2$}
\date{%
    $^1$ Department of Psychiatry \& Behavioral
Sciences, Memorial Sloan Kettering Cancer Center
    \\%
    $^2$
    Department of Epidemiology \& Biostatistics, MSKCC
    \\[2ex]%
    \today
}

\begin{document}
\maketitle

\section{Purpose}
This document summarizes the ``varying-intercepts, varying-slopes''
model in this example.
\section{Problem} The model is motivated by a long-standing challenge in cognitive 
aging research, seen here from a graph in Salthouse (2019).~\footnote{\scriptsize Salthouse (2019). \emph{Psychology and Aging.} \url{http://dx.doi.org/10.1037/pag0000288}}
  \begin{center}
  \includegraphics[width=0.50\textwidth]{Salthouse_Traj_Aging18_Fig1.pdf}
  \end{center}
If you analyze cognitive assessments
cross-sectionally using each person's very first assessment (open
circles above), you typically get a cognitive aging trend
that shows nearly linear decline with older age.
However, if you analyze the longitudinal data over
repeated assessments (filled circles), you often get an
increasing trend over time.  Thus, the main problem is 
that repeated assessments are confounded by 
the \emph{test exposure} effect (aka
`practice effect').  For pragmatic purposes we should
call it practice effect throughout this tutorial. 

The cross-sectional data is not affected by practice
effect and thus may better explain the true aging trend.
However, the cross-sectional aging trend has its own
problems. It is affected by the cohort effect, i.e.,
people born in the 1940's (in their 70's to 80's by 2023)
may be different than younger people born in the 1980's.
Details on the varous issues are not immediately relevant
to our model, but they can be found in the Salthouse
paper.  

In this tutorial, we use a Bayesian varying-intercepts,
varying-slopes model to explain both the cross-sectional
aging trend while \emph{simultaneously} fit the
longitudinal practice effect. You don't have to analzye
cross-sectional and longitudinal data separately.

Before we proceed to data modeling, we should get an
intuition on why the model involves the intercepts and
slopes. This graph shows the neurocognitive assessments
from four randomly-selected cancer survivors and 
controls. 
\begin{center}
\includegraphics[width=0.50\textwidth]{varInt_varSlp.pdf}
\end{center} 
From each person's 4 assessments, we can fit a personal
regression line in blue.  Here we show one control participant's
regression with an intercept of $\alpha = 0.7$ and a
slope of $\beta=0.004$.  The intercept of $\alpha = 0.7$
represents the person's very first assessment that is
presumably free of practice effect.  This person scored 0.70
standard deviations above the norm at study enrollment.
The slope of $\beta = 0.004$
represents this person's expected rate of change per month, and
practice effect is subsumed in this slope.  Notice how the
practice effect is now separated from 
cross-sectional aging.

\section{Preliminary Model}

We can now turn to a simpler version of the full model.  
The preliminary model is divided into
2 levels.  In level 1, each person's 4 longitudinal
assessments are first distilled into an intercept and a
slope. Here we use $i$ to index persons and $t$ to index
assessment time points.  
\begin{equation}
  y_{i[t]} \sim \mathrm{N}\bigl(\alpha_{i} + 
        \beta_{i} \mathrm{Months}_{i[t]}, \sigma^2_{\epsilon}\bigr),\ 
    \mathrm{for}\ i = 1, \dots, n;\quad  t = 1, \dots, 4,
    \label{EQ:Y}
\end{equation}
where the assessment collected from the $i$th person at
time $t$ is shown as $y_{i[t]}$.  The square brackets
show that the $t$th assessment is nested within the $i$th
person.
The varying intercepts and varying slopes are further
analyzed in level 2, where the intercepts $\alpha_i$ and slopes 
$\beta_i$ are
modeled by covariates.
\begin{equation}
\begin{aligned}
        \alpha_i &=
        \gamma_{00} +
        \gamma_{01} \mathrm{Age}_i + 
        \gamma_{02} \mathrm{Age}^2_i + 
        \gamma_{03} \mathrm{survivor}_i + \\
        & \qquad \qquad 
        \gamma_{04} \mathrm{survivor}_i\cdot \mathrm{Age}_i+ 
        \gamma_{05} \mathrm{survivor}_i\cdot \mathrm{Age}^2_i 
        \\ 
        \beta_i &= 
        \gamma_{10} + \gamma_{11} \mathrm{survivor}_i + 
        \gamma_{12} \mathrm{age4Q}_i + %
        \gamma_{13} \mathrm{survivor}\cdot
	\mathrm{age4Q}_i,
	\label{EQ:ALPHA:BETA}
\end{aligned}
\end{equation}
the intercepts being a 
quardratic function of age at enrollment.  More specifically,
the control cohort follows this quadratic aging trend:
$
\gamma_{00} + \gamma_{01} \mathrm{Age}_i + \gamma_{02}
\mathrm{Age}^2_i
$
and the survivor cohort differs from
the controls by 
$
\gamma_{03} \mathrm{survivor}_i +
        \gamma_{04} \mathrm{survivor}_i\cdot \mathrm{Age}_i+ 
        \gamma_{05} \mathrm{survivor}_i\cdot
	\mathrm{Age}^2_i.
$
Additionally, the intercepts $\beta_i$ 
are modeled as a function of 4
age quartiles in the $\mathrm{age4Q}_i$ predictor
$
        \beta_i = 
        \gamma_{10} + \gamma_{11} \mathrm{survivor}_i + 
        \gamma_{12} \mathrm{age4Q}_i + %
        \gamma_{13} \mathrm{survivor}\cdot
	\mathrm{age4Q}_i.
$
This effectively fits 8 separate practice effects, 4 for
each of the age quartiles in controls and 4 for survivors.
Finally, we gather
equations~\eqref{EQ:Y} -- \eqref{EQ:ALPHA:BETA} together to yield
the following 2-level model:
\begin{equation}
\begin{aligned}
\mathrm{Level 1:} \phantom{2cm} & \\
  y_{i[t]} &\sim \mathrm{N}\bigl(\alpha_{i} + 
        \beta_{i} \mathrm{Months}_{i[t]}, \sigma^2_{\epsilon}\bigr),\ 
    \mathrm{for}\ i = 1, \dots, n;\quad  t = 1, \dots, 4,
    \\
\mathrm{Level 2:} \phantom{2cm} & \\
        \alpha_i &=
        \gamma_{00} +
        \gamma_{01} \mathrm{Age}_i + 
        \gamma_{02} \mathrm{Age}^2_i + 
        \gamma_{03} \mathrm{survivor}_i + \\
        & \qquad \qquad 
        \gamma_{04} \mathrm{survivor}_i\cdot \mathrm{Age}_i+ 
        \gamma_{05} \mathrm{survivor}_i\cdot \mathrm{Age}^2_i 
        \\ 
        \beta_i &= 
        \gamma_{10} + \gamma_{11} \mathrm{survivor}_i + 
        \gamma_{12} \mathrm{age4Q}_i + %
        \gamma_{13} \mathrm{survivor}\cdot
	\mathrm{age4Q}_i,
	\label{EQ:HLM}
\end{aligned}
\end{equation}
Readers may recognize this model as identical to how
Bryk and Raudenbush would specify a 2-level HLM model.
This is among the many ways of writing the basic
multilevel model.  You can plug $\alpha_i$ and $\beta_i$
back into equation~\eqref{EQ:Y} and re-express the
2-level model in one line:
$$
\begin{aligned}
  y_{i[t]} &\sim \mathrm{N}\bigl(\alpha_{i} + 
        \beta_{i} \mathrm{Months}_{i[t]},
	\sigma^2_{\epsilon}\bigr), \\
  &\sim \mathrm{N}\bigl(
        \gamma_{00} +
        \gamma_{01} \mathrm{Age}_i + 
        \gamma_{02} \mathrm{Age}^2_i + 
        \gamma_{03} \mathrm{survivor}_i + \\
        & \qquad \qquad 
        \gamma_{04} \mathrm{survivor}_i\cdot \mathrm{Age}_i+ 
        \gamma_{05} \mathrm{survivor}_i\cdot
	\mathrm{Age}^2_i + \\
	& \qquad \qquad [
        \gamma_{10} + \gamma_{11} \mathrm{survivor}_i + 
        \gamma_{12} \mathrm{age4Q}_i + %
        \gamma_{13} \mathrm{survivor}\cdot
	\mathrm{age4Q}_i
	] \cdot
        \mathrm{Months}_{i[t]},
	  \sigma^2_{\epsilon} \bigr).
\end{aligned}
$$
This one-line model can be easily fitted using the R
package \texttt{lme4}, SAS PROC MIXED, the HLM package,
or Mplus.  Using \texttt{lme4}, for example, the model
can be fitted like this:
\begin{verbatim}
lmer(y ~ age + I(age^2) + survivor + survivor:age +
    survivor:I(age^2) + months + survivor:months +
    age4Q:months + survivor:age4Q:months + (1 + months |
    study_id))
\end{verbatim}
Note that the R syntax matches the math equation.
The Bayesian
model is exactly the same as a conventional, HLM-like
expression.   Next we will express the model in a Bayesian
way, with priors and all.  The differences
should become more apparent. 

\section{Bayesian Model}

The Bayesian expression has the same level-1 model.  In
level 2, however, the varying intercepts and slopes are
drawn from a bivariate normal distribution with means
$\mu$ and covariance $\Omega$.  
\begin{equation}
\begin{aligned}
\mathrm{Level 1:} \phantom{2cm} & \\
        y_{i[t]} &\sim \mathrm{N}\bigl(\alpha_{i} + 
        \beta_{i} \mathrm{Months}_{i[t]}, \sigma^2_{\epsilon}\bigr),\ 
    \mathrm{for}\ i = 1, \dots, n;\quad  t = 1, \dots, 4  \\
\mathrm{Level 2:} \phantom{2cm} & \\
    \begin{pmatrix}
    \alpha_{i}\\
    \beta_{i}
    \end{pmatrix} &\sim \mathrm{N} \Biggl(
      \begin{pmatrix}
       \hat\mu_{\alpha_i}\\
       \hat\mu_{\beta_i}
      \end{pmatrix}\! \!,
      \Omega  
      \Biggr), \\
    \mathrm{where}& \\
        \hat\mu_{\alpha_i} &=
        \gamma_{00} +
        \gamma_{01} \mathrm{Age}_i + 
        \gamma_{02} \mathrm{Age}^2_i + 
        \gamma_{03} \mathrm{survivor}_i + \\
        & \qquad \qquad 
        \gamma_{04} \mathrm{survivor}_i\cdot \mathrm{Age}_i+ 
        \gamma_{05} \mathrm{survivor}_i\cdot \mathrm{Age}^2_i 
        \\
        \hat\mu_{\beta_i} &= 
        \gamma_{10} + \gamma_{11} \mathrm{survivor}_i + 
        \gamma_{12} \mathrm{age4Q}_i + %
        \gamma_{13} \mathrm{survivor}\cdot \mathrm{age4Q}_i \\
        \Omega &= \begin{pmatrix}
               \sigma_\alpha & 0           \\
               0             & \sigma_\beta
               \end{pmatrix} 
               \boldsymbol{R} 
               \begin{pmatrix}
               \sigma_\alpha & 0           \\
               0             & \sigma_\beta
               \end{pmatrix}  \\
        \boldsymbol{R} &\sim \mathrm{LKJCorr}(2) \\
        \boldsymbol{\gamma} &\sim\mathrm{Normal}(\gamma, \sigma_{\gamma})\\
        \gamma &\sim \mathrm{Normal}(0, 10) \\
        \sigma_{\gamma} &\sim \mathrm{HalfCauchy}(0, 1)
          % &sigma^2_{\mathrm{cluster}}), \ \mathrm{for}\ j = 1, \dots, J,
  \label{EQ:BAYES} 
\end{aligned}
\end{equation}
The notation using $\hat\mu_{\alpha_i}$
and $\hat\mu_{\beta_i}$ is a different way of
expressing the level-2 equations.  The contents are the
same as in equation~\eqref{EQ:HLM}.

The covariance matrix $\omega$
is constructed by factoring
it into separate standard deviations 
$\sigma_{\alpha_i}, \sigma_{\beta_i}$ and 
a correlation matrix $\boldsymbol{R}$.  
The variances
$\sigma_{\alpha_i}, \sigma_{\beta_i}$ are
for the intercepts and slopes in the diagonals, and a
correlation rho between intercepts and slopes. A negative
correlation would suggest that a high intercept is
associated with a slower growth and vice versa.
The correlation matrix $\boldsymbol{R}$
follows an LKJ
prior,\url{https://distribution-explorer.github.io/multivariate_continuous/lkj.html},
which offers more flexibility than
the Wishart distribution.
An LKJCorr(2) prior is essentially expressing the prior
belief of a weak correlation coefficient mostly between -0.5 and
+0.5, but it can also be outside that
range.\url{https://psychstatistics.github.io/2014/12/27/d-lkj-priors/}
The remaining lines define fixed priors and hyper-priors
for the parameters. 
The correlation between the intercepts and slopes 
is a noteworthy feature of the model because we can
estimate the correlations of these random effects,
thereby providing
information on the co-occurrences between the
cross-sectional aging trend and practice effect.  This
correlation is omitted when they are analyzed separately.  

\section{Summary}
This completes the model equations.  The Bayesian
approach is similar to the HLM model, but it contains
additional information on the covariance matrix and
priors on the model coefficients.

\end{document}
